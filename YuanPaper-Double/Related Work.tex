\section{Related Work}

In this chapter, we will review previous the energy management and routing schemes for satellite network. 


\label{PRC}

\subsection{Energy Efficiency in Satellites}
\cite{BATTERYTYPE} proposed a design method such as, quantity and type, for solar panels and batteries to ensure that satellites that are launched and offline can perform all tasks without lack of power until the end of their life. In addition, under power management, they first classify the tasks to be performed by the satellites according to QoS. Then, without affecting the sufficiency of the power supply, they proposed a way to make the connected satellites perform the tasks as high as possible QoS version. First of all, in terms of design, they can reduce production costs and allocate more weight and space for effective payloads. In terms of management, they can maximize the task performance of a given power system design.

For another, more studies work on the management of missions to make use of energy more efficiently. Gatzianas et al. \cite{CROSSLAYER} proposed a cross-layer resource allocation scheme for wireless networks operating with rechargeable batteries, aiming to maximize total system utility.

Fu et al. \cite{ADMISSIONCONTROL} focused on the communication satellites with limited energy. They redesigned an energy allocation and admission control scheme to maximize the reward. 

\subsection{Routing in Satellite Networks}
\cite{INCOME} proposed using distributed routing scheme based on every ISL information like propagation delay, packet loss ratio, bandwidth and buffer length then use Shapley value to calculate the income distribution plan. Each satellite has its starting point to increase its maximum profit, that is, a satellite with a stronger processing capability to do the forwarding. \cite{LCD1} adopt hop-by-hop distributed routing schemes, and made rules for satellite to choose the next-hop without complex calculation. The main idea is basically trying to transmit in the same orbit to the same latitude of the destination satellite, performing cross-orbit transmission, and then making some changes according to the polar circle and seams in the rules. However, it does not consider the possible congestion of the inter-satellite link or the problem of satellite power because the satellite does not calculate other information. \cite{LCD2} and \cite{LCD3} are basically follow with same idea. However, still lots of study work on centralized routing scheme, because the movement of satellite is regular. \cite{CHARACTERISTICS} uses the movement characteristics of the satellite, the remaining time of each link can be calculated, combined with the path delay. If the link remaining time is less than the path delay, the routing table in the satellite node could be changed to the next handover satellite in advance, and the data packets sent on the link which is going to be changed will be stopped, thereby avoiding the packets lost caused by the switching in the link. \cite{LOWTRACK} calculated the propagation delay, packet loss ratio, and bandwidth between links, finding the candidate paths according to the upper limit of hops, and then use the GWO algorithm to find the best route. \cite{PRIORITY} proposed an operation center on the ground calculates the routes according to the priority, bandwidth, remaining time of links. And it creates not only just one route but backup routes to solve link failure or congestion. But these routing schemes, either centralized or distributed, didn’t consider the energy of the satellite. Even if we don't consider the lifetime of the constellation, it is possible when the data sent to a satellite that lacks energy.


\subsection{Energy-Efficient Routing in the LEO network}

\cite{LOCATIONAWARE} proposed a concept of the satellite's location with sun period and eclipse period, it first computes all the possible routes and all the propagation delay of inter-plane ISLs and intra-plane ISLs. Then it sums up all propagation delay of links whose transmitter satellites are in the eclipse period. The cost of a route is computed by two main delays: the sum of all propagation delays and the sum of propagation delays in the eclipse period. And there will be a weighting factor to determine which part the user pays more attention to. If the weighting factor of the first part is bigger, the route computation will focus on finding the route which propagation delay is small, otherwise, the route computation will focus on saving energy. Thus the algorithm can provide a trade-off between the operational lifetime of the satellite networks and the propagation delay of the selected routes. By sacrificing some propagation delay, this method can achieve a kind of load balance. 
\cite{TOWARDS} proposed an algorithm that computes link costs iteratively to compute a routing that minimizes the total recharge/discharge cycle number. It can also turn the satellite which is not needed temporarily into sleep mode, which means the satellite will shut down partial power for some components temporarily. \cite{QOS} thought that service should be classified to make the best use of the network utility, and the algorithm also brings the concept of energy to make sure any reliable data has been sent. Nevertheless, all these schemes are centralize routing, and if any link failure occurs, the route cannot be updated immediately.