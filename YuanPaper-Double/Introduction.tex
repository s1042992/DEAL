\section{Introduction}
\label{ch:Introduction}
Satellite networks can provide coverage in all places on the earth, even the oceans, and deserts. Without terrain restrictions, communication between two cities becomes much more accessible through satellite technology. Thus people are getting more and more interest in satellite networks than terrestrial networks in a decade. According to different heights of orbit, we can divide satellites into three types: geostationary-earth orbit (GEO) satellite, medium earth orbit (MEO) satellite systems, and low-earth orbit (LEO) satellite. Compared to GEO satellite systems and MEO satellite systems, LEO satellite systems have higher throughput, higher data rate, and a lower orbit altitude, providing lower propagation delay and more possibility to achieve real-time communications\cite{AGENTBASED}. Therefore, researching LEO networks becomes popular in the field of satellite recently.

\subsection{Limits of LEO satellite}
Communication devices in satellites are typically powered by solar panels and battery cells, which are carefully designed to guarantee power supply and avoid deficiency\cite{CISCO}. When a LEO satellite spins around the earth, it will be covered periodically by the earth's shadow. Thus we can divide the LEO satellite orbital period into the solar period and eclipse period. When the sunlight can irradiate the satellite, it means the satellite is in the solar period. Otherwise, when the satellite is in the earth’s shadow, it means the satellite is in the eclipse period. When a satellite is in the eclipse period, the satellite’s energy consumption is relatively large. The battery is the only power supplier when a satellite is in the eclipse period. In contrast, the satellite is able to utilize solar energy directly in the solar period.

A battery's lifetime is affected by the depth of discharge(DoD). In many types of batteries, the total energy stored in the battery cannot be fully discharged without causing severe and often irreparable damage to the battery. Numerous cycles of use and more charge demanded on each cycle would lead to a shorter battery's lifetime. Moreover, the charge and discharge frequency of the LEO satellite is much higher than the frequency of the GEO satellite per day. The LEO satellite has an orbital period of 100 minutes. The eclipse period is 30–40 minutes per orbit, and it means that the LEO satellites can only use the battery's energy without charging for 30 to 40 minutes per cycle\cite{DOOR}. It may lead to an exceedingly low percentage of the battery's capacity. As far as the network lifetime is concerned, considering the DoD of the LEO satellite is necessary.

For instance, when we discharge the battery from 100\% to 20\% and then recharge back to 100\% which DoD is 80\%, the battery life cycle will be smaller than the battery discharged from 100\% to 70\%, and then recharge back to 100\% which DoD is 30\% \cite{TOWARDS}

Managing the power of a satellite is quite essential in a satellite network. If a battery runs out of cycles, the satellite will no longer work. Not only constructing and launching a satellite is very expensive, but exchanging batteries on satellites is also difficult \cite{LAUNCH}. People want to extend the life of a satellite to provide the service and lower the cost as much as possible.

\subsection{Benefits of saving satellite's energy}
Attaching importance to LEO's energy management is significant because it could be translated to cost-saving. The satellite, which needs lower energy, requires a smaller battery, smaller solar panels. Lighter hardware on the satellite would bring about saving costs on launching, and it will provide economic benefits. However, existing energy-efficient routing approaches on the Internet cannot be applied straightforwardly in satellite networks. It is necessary to understand these issues comprehensively and understand how network routings affect the lifetime of satellite networks. 

\subsection{Routing on satellite network}
Different from the terrestrial networks, the topology of constellation networks is usually sparse and deployed regularly.  and every satellite in this kind of network will cover plenty of users. Considering the un-uniform user distribution on earth surface, satellites in different orbit positions will encounter distinct density level. As a result, the traffic density will be distributed unbalanced in the satellite network.

A routing algorithm is a set of step-by-step operations used to direct Internet traffic efficiently. When the data leaves its source, many different paths it can take to its destination. Depending on whether the decision changes or not with time, the algorithm can be static or dynamic. In a satellite network, only dynamic solutions are meaningful because the high-speed movement of LEO satellites results in a dynamic change in satellite network topology.

A tranditional way to route on the LEO satellite network is to compute all the paths on the ground station centrally and then broadcast the information to all the satellites in the constellation. The satellites forward the data based on the routing table originally from the ground station. This centralized approach would be possible to find the optimal solution. However, this solution with a limited number of ground stations is challenging. The routing tables must be updated and sent to the satellites frequently. These updates require the ground-satellite link (GSL), but its intermittent availability can be insufficient to track the topology changes.

In addition, the total bandwidth has to be shared with other data. If the ground station frequently sends the routing information to the satellite to keep up with the topology change, it will occupy a specific bandwidth of GSL and ISL in the long term. Therefore, computing routes on the satellite itself is an approach to avoiding occupying the bandwidth of GSL

In addition, the total bandwidth has to be shared with other data. If the ground station frequently sends the routing information to the satellite to keep up with the topology change, it will occupy a specific bandwidth of GSL and ISL in the long term. Therefore, computing routes on the satellite itself is an approach to avoiding occupying the bandwidth of GSL.

Distributed routing refers to that every node on the network has the ability to decide the route of themselves and also requires the information of decision from each other. A simpler way for LEO constellation is a hop-by-hop autonomous routing. Each satellite could choose the next hop from the list of available links to route the packet in the best direction towards the destination. There is two motivation for this approach in the LEO network: first, increasing the resilience of the routing, with satellites able to run the routing algorithm independently of the rest of the network; on the other hand, exploiting the predictability of a LEO topology, where nodes do not move randomly, but according to well-known physical laws. The predictability allows skipping the "Hello" messages utilized to discover adjacent nodes in terrestrial ad-hoc networks. It minimizes the control information and overheads. Taking local decisions at each satellite also facilitates the management of congestion and failures in the network. A distributed algorithm can be run alone or used as a backup solution of more traditional routing table-based solutions.

A routing policy that uses only a single path is not necessarily optimal from the viewpoint of an entire network even though the path is the shortest one.
In order to balance the traffic load in the whole satellite network, most load balancing routing algorithm will collect link and traffic load states information and change/split the path for packets
and the traffic splitting mechanism plays an important role in traffic load balancing. Thus, in our scheme, every node in the network can split the traffic flow flexibly according to their neighbors' status such as traffic density, remaining energy, and the instability of ISLs to distribute the traffic, avoiding congestion.